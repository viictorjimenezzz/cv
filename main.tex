%%%%%%%%%%%%%%%%%%%%%%%%%%%%%%%%%%%%%%%%%
% Medium Length Professional CV
% LaTeX Template
% Version 2.0 (8/5/13)
%
% This template has been downloaded from:
% http://www.LaTeXTemplates.com
%
% Original author:
% Rishi Shah 
%
% Important note:
% This template requires the resume.cls file to be in the same directory as the
% .tex file. The resume.cls file provides the resume style used for structuring the
% document.
%
%%%%%%%%%%%%%%%%%%%%%%%%%%%%%%%%%%%%%%%%%

%----------------------------------------------------------------------------------------
%	PACKAGES AND OTHER DOCUMENT CONFIGURATIONS
%----------------------------------------------------------------------------------------

\documentclass{resume} % Use the custom resume.cls style

\usepackage[left=0.75in,top=0.6in,right=0.75in,bottom=0.6in]{geometry} % Document margins
\newcommand{\tab}[1]{\hspace{.2667\textwidth}\rlap{#1}}
\newcommand{\itab}[1]{\hspace{0em}\rlap{#1}}
\name{V\'ictor Jim\'enez Rodr\'iguez} % Your name
\address{Zurich, Switzerland $\big|$ Barcelona, Spain} % Your address
%\address{123 Pleasant Lane \\ City, State 12345} % Your secondary addess (optional)
\address{(+41) 076 264 58 81 $\big|$ victorjimenezrodriguez00@gmail.com} % Your phone number and email

\begin{document}
\setlength{\baselineskip}{-0.2em}

\begin{rSection}{Summary}
    Machine learning researcher with a background in physics and statistics, and a proven track record in 
    both academic and industrial environments. Interested in the assessment and understanding of robustness
    in machine learning systems, with experience in the development and deployment of agentic retrieval-augmented generation (A-RAG)
    pipelines in high-stakes applications.
\end{rSection}

%----------------------------------------------------------------------------------------
%	EXPERIENCE SECTION
%----------------------------------------------------------------------------------------


\begin{rSection}{Experience}

    {\bf Uthereal (ETH AI Center Startup)} -- Zurich
    \\ \textit{AI Scientist} \hfill {\em Since Jan. 2025} \\

    Coordination and contribution to multiple R\&D projects involving the integration of A-RAG pipelines in high-risk domains (medical, legal, technical...).
    Collaboration with academic partners for joint research initiatives and grant applications.

    \textit{ML Engineer} \hfill {\em Jan. 2024 -- Dec. 2025}
    
    Designed and implemented an end-to-end agentic RAG system with advanced document understanding 
    and explainability features, alongside a comprehensive experimental framework for systematic evaluation 
    and deployment.

    % \begin{itemize}
        % \item Developed a full-scale agentic information retrieval system (A-RAG), integrating OCR, 
        % document feature engineering, keyword modeling, knowledge graph generation, hybrid embedding search,
        % chain-of-thought reasoning, and explainability. \\
        % % \item Contributed to an industrial-academic partnership Innosuisse grant application, focusing on 
        % % the integration of A-RAG in the medical domain. The grant was awarded CHF 980,000.
        % % \item Designed and co-supervised thesis and semester projects for ETH master's students, including 
        % % research initiatives linked to the Innosuisse grant.\
        % \item Built an end-to-end ML experimental framework for A-RAG pipeline evaluation and benchmarking. 
        % Integrated Pydantic framework with configuration-based instantiation, AWS deployment, 
        % and W\&B logging for streamlined experimentation and reproducibility. \\
    % \end{itemize}

    {\bf Institute for Machine Learning (ETH)} -- Zurich \hfill {\em Apr. 2025 -- Oct. 2025} 
    \\ Statistical Machine Learning -- Prof. Dr. Fanny Yang \\
    \\ \textit{Research internship} \\

    Research project focused on the derivation of finite-sample robustness guarantees 
    in high-dimensional compositional settings under arbitrarily large subpopulation shifts.

    {\bf Institute for Machine Learning (ETH)} -- Zurich \hfill {\em Nov. 2023 -- Sep. 2024} 
    \\ Information Science and Engineering -- Prof. Dr. Joachim M. Buhmann \\
    \\ \textit{Research traineeship} \\

    Master's thesis with honors: \textit{Improved robustness of deep learning models through posterior agreement based
    model selection.} Manuscript derived from the thesis has been submitted for publication.
    
    {\bf Department of Physics (TUM)} -- Munich \hfill {\em Feb. 2022 -- Oct. 2022} 
    \\ Physics of Energy Conversion and Storage -- Prof. Dr. Aliaksandr Bandarenka \\
    \\ \textit{Research traineeship} \\

    Bachelor's thesis with honors: \textit{EIS characterization of lithiated TiO\textsubscript{2}-coated LICGC electrolytes for 
        the stabilization of the SEI in all-solid-state lithium batteries.} Contributed to published work: \textit{Bandarenka et al., ChemSusChem 2024, e202401026, DOI: 10.1002/cssc.202401026.}
    
\end{rSection}


%----------------------------------------------------------------------------------------
%	EDUCATION SECTION
%----------------------------------------------------------------------------------------

\begin{rSection}{Education}
{\bf Master's degree in Statistics and Operations Research} \hfill {\em 2022 -- 2024} 
\\ Facultat de Matem\`atiques i Estad\'istica -- UPC (Barcelona) \vspace{0.5em} \\
Completed track in statistical inference, optimization theory, and machine learning. [9.05/10]

% \begin{itemize}
%     \item Honors in Machine Learning, Bayesian Analysis and Programming.
%     \item Completed track in statistical learning, optimization theory and statistical inference.
%     \end{itemize}

{\bf Bachelor's degree in Engineering Physics} \hfill {\em 2018 -- 2022} 
\\ ETSETB -- UPC (Barcelona) \vspace{0.5em} \\ 
Elective coursework included computational electromagnetism, advanced materials, simulation 
of condensed matter, quantum optical technologies, photonics, and computational biophysics. 
Engineering courses covered control theory, circuit theory, signal processing, and antenna design.\\

{\bf Scientific-Technological Baccalaureate} \hfill {\em 2016 -- 2018} 
\\ Maristes Sants-Les Corts (Barcelona) \vspace{0.5em} \\
Ranked in the top 0.1\% of PAU exams, securing a full-tuition scholarship for the first year of studies.

\end{rSection}
%----------------------------------------------------------------------------------------
%	TECHNICAL STRENGTHS SECTION
%----------------------------------------------------------------------------------------

% \begin{rSection}{Technical Skills}

% \begin{tabular}{ @{} >{\bfseries}l @{\hspace{6ex}} l }
% Language \ & C, C++ \\
% Software  \ & MS Office, Latex \\
% \end{tabular}

% \end{rSection}

%----------------------------------------------------------------------------------------
%	LANGUAGES
%----------------------------------------------------------------------------------------

\newpage

\begin{rSection}{Languages}
    \begin{tabular}{ @{} >{\bfseries}l @{\hspace{6ex}} l }
    Catalan, Spanish \ & Native \\
    English  \ & Proficient \\
    German  \ & Intermediate \\
    \end{tabular}
\end{rSection}

\begin{rSection}{Technical Skills}
    \begin{tabular}{ @{} >{\bfseries}l @{\hspace{6ex}} l }
    Python \ & Machine learning, data analysis, computational physics. \\
    UNIX/Linux \ & Command-line operations, file management, system configuration in Ubuntu. \\
    HPC systems \ & Experience with Euler (CSCS): job scheduling, parallel computing. \\
    R  \ & Statistical inference, statistical learning. \\
    MATLAB  \ & Numerical methods for mathematics, physics and engineering. \vspace{-0.3em} \\
            & Includes dynamical systems, FEM analysis, signal processing, IPM optimization. \\
    AMPL \ & MILP optimization, large-scale optimization, stochastic programming. \\
    Stan \ & Bayesian analysis. \\
    Scala \ & FOOP, Spark RDDs. \\
    SAS \ & Statistical data analysis. \\
    Fortran \ & Molecular dynamics and Monte-Carlo simulations. \\
    C, C\texttt{++} \ & Analog and digital circuit control. \\
    \end{tabular}
\end{rSection}
   
%----------------------------------------------------------------------------------------
%	EXAMPLE SECTION
%----------------------------------------------------------------------------------------

\begin{rSection}{Other projects}
% {\bf Erasmus ULISSES Project Ideathon} -- UL (Lisbon) \hfill \vspace{0.5em} {\em Jul. 2023} 
% \\ Three-week intensive ideathon aimed at addressing an ocean sustainability
% challenge. The project required a previous semester e-learning
% phase and a thorough investigation of the state-of-the art in marine
% sciences. Ideathon involved satellite data collection, solution design, viability assessment and product showcasing processes. \\
% % (Copernicus, IOOS, EMODnet...)

% {\bf Modelling and Design of a Paul Ion Trap} -- Computational Electromagnetism \hfill \vspace{0.5em} {\em 2021} 
% \\ Numerical simulation of the distribution of charges and the potential generated by a quadrupole electrode
% configuration using the method of moments, resulting in confined ion trajectories for harmonic potentials of varying frequencies.
% Starting with simpler models in 2D and 3D, the project provides optimal design parameters needed for
% confinement depending on the number and nature of the ions.\\

% {\bf Calcium-mediated regulation of astrocytes response} -- Computational biophysics \hfill \vspace{0.5em} {\em 2021} 
% \\ Analysis of the $Ca^{2+}$ concentration in brain astrocytes under different inositol triphosphate (IP3)
% signaling dynamics. Hodgkin-Huxley model was used to show spike-stimulated calcium oscillations due to
% the production of IP3 reaching threshold concentrations, and the continued influx of ions when neuron-coupled dynamics
% are considered, a malfunction that has been linked to the appearance of several mental diseases.
% \\

% {\bf Erasmus ULISSES Ideathon} -- UL (Lisbon) \hfill  {\em Jul. 2023} \\
% Three-week intensive ideathon aimed at addressing an ocean sustainability challenge.

{\bf Lasso and Bayes: a demostration using real estate market data} \hfill {\em 2023} 
\\ Bayesian analysis. \\

{\bf Modelling and design of a Paul Ion Trap} \hfill {\em 2021} 
\\ Computational electromagnetism, finite element method, EM momentum method.\\

{\bf Calcium-mediated regulation of astrocytes response in the brain} \hfill {\em 2021} 
\\ Computational biophysics, dynamical systems modelling. \\

{\bf Design and implementation of a sound recorder, processor and player in the electronics laboratory.
TD-PSOLA and Phase Vocoder algorithms in a STM32 
% NUCLEO-F303RE
 microprocessor.} \hfill {\em 2021}
\\ Signal processing, analog and digital circuits, PWM conversion. \\

{\bf BB84 guarantees for QKD in free-space-link communication systems} \hfill {\em 2018} 
\\ Quantum physics, cryptography, free-space-link communication. \\
\end{rSection}

\end{document}