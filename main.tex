%%%%%%%%%%%%%%%%%%%%%%%%%%%%%%%%%%%%%%%%%
% Medium Length Professional CV
% LaTeX Template
% Version 2.0 (8/5/13)
%
% This template has been downloaded from:
% http://www.LaTeXTemplates.com
%
% Original author:
% Rishi Shah 
%
% Important note:
% This template requires the resume.cls file to be in the same directory as the
% .tex file. The resume.cls file provides the resume style used for structuring the
% document.
%
%%%%%%%%%%%%%%%%%%%%%%%%%%%%%%%%%%%%%%%%%

%----------------------------------------------------------------------------------------
%	PACKAGES AND OTHER DOCUMENT CONFIGURATIONS
%----------------------------------------------------------------------------------------

\documentclass{resume} % Use the custom resume.cls style

\usepackage[left=0.75in,top=0.6in,right=0.75in,bottom=0.6in]{geometry} % Document margins
\newcommand{\tab}[1]{\hspace{.2667\textwidth}\rlap{#1}}
\newcommand{\itab}[1]{\hspace{0em}\rlap{#1}}
\name{V\'ictor Jim\'enez Rodr\'iguez} % Your name
\address{Zurich, Switzerland $\big|$ Barcelona, Spain} % Your address
%\address{123 Pleasant Lane \\ City, State 12345} % Your secondary addess (optional)
\address{(+41) 076 264 58 81 $\big|$ victorjimenezrodriguez00@gmail.com} % Your phone number and email

\begin{document}
\setlength{\baselineskip}{-0.2em}

\begin{rSection}{Summary}
    Machine learning engineer with a background in physics and statistics, and a proven track record in 
    both academic and industrial environments.
    Currently looking for a research internship position to contribute to exciting projects in machine learning 
    and computational science.
\end{rSection}

%----------------------------------------------------------------------------------------
%	EXPERIENCE SECTION
%----------------------------------------------------------------------------------------


\begin{rSection}{Experience}
    {\bf Uthereal} -- ETH AI Center (Zurich) \hfill {\em Since Jan. 2024} 
    \\ Machine learning engineer \\
    \begin{itemize}
        \item Developed a full-scale agentic information retrieval system (A-RAG), integrating OCR, 
        document feature engineering, keyword modeling, knowledge graph generation, hybrid embedding search,
        chain-of-thought reasoning, and explainability. \
        \item Contributed to an industrial-academic partnership Innosuisse grant application, focusing on 
        the integration of A-RAG in the medical domain. The grant was awarded CHF 980,000.
        \item Designed and co-supervised thesis and semester projects for ETH master's students, including 
        research initiatives linked to the Innosuisse grant.\
        \item Built an end-to-end ML experimental framework for A-RAG pipeline evaluation and benchmarking. 
        Integrated Pydantic framework with configuration-based instantiation, AWS deployment, 
        and W\&B logging for streamlined experimentation and reproducibility. \
    \end{itemize}

    {\bf Institute for Machine Learning} -- ETH (Zurich) \hfill {\em Nov. 2023 -- Sep. 2024} 
    \\ Research traineeship -- Prof. Dr. Joachim M. Buhmann \\ \\
    \\ \textit{Improved robustness of deep learning models through posterior agreement based 
    model selection.}
    
    Master's thesis with Honors. Candidate for the ETH Medal in CS. Manuscript derived from the thesis currently under submission for publication.

    {\bf Physics of Energy Conversion and Storage} -- TUM (Munich) \hfill {\em Feb. 2022 -- Oct. 2022} 
    \\ Research traineeship -- Prof. Dr. Aliaksandr Bandarenka \\
    \\ \textit{EIS characterization of lithiated TiO\textsubscript{2}-coated LICGC electrolytes for 
    the stabilization of the SEI in all-solid-state lithium batteries.}
    
    Bachelor's thesis with Honors. Contributed to published work: 
    \textit{Characterization of the Lithium/Solid Electrolyte Interface in the Presence of Nanometer-thin TiOx Layers for All-Solid-State Batteries -- A. Bandarenka, et al. ChemSusChem 2024, e202401026.}
\end{rSection}


%----------------------------------------------------------------------------------------
%	EDUCATION SECTION
%----------------------------------------------------------------------------------------

\begin{rSection}{Education}
{\bf Master's degree in Statistics and Operations Research} \hfill {\em 2022 -- 2024} 
\\ Facultat de Matem\`atiques i Estad\'istica -- UPC (Barcelona) \vspace{0.5em} \\
Completed track in statistical inference, optimization theory and machine learning. [9.05/10]

% \begin{itemize}
%     \item Honors in Machine Learning, Bayesian Analysis and Programming.
%     \item Completed track in statistical learning, optimization theory and statistical inference.
%     \end{itemize}

{\bf Bachelor's degree in Engineering Physics} \hfill {\em 2018 -- 2022} 
\\ ETSETB -- UPC (Barcelona) \vspace{0.5em} \\ 
Elective coursework included computational electromagnetism, advanced materials, numerical simulation 
of condensed matter, quantum computing and optical technologies, photonics, and computational biophysics. 
Engineering courses covered control theory, signal processing, and antenna design.\\
% \begin{itemize}
% \item Elective coursework included computational electromagnetism, advanced materials, 
% numerical simulation of condensed matter, quantum computing and optical technologies, photonics and 
% computational biophysics.
% \item Engineering coursework included control theory, signal processing, 
% antennas and transmission lines.
% \end{itemize}
% \\ Elective coursework included computational electromagnetism, advanced materials, 
% numerical simulation of condensed matter, quantum computing and optical technologies, photonics and 
% computational biophysics. \\
% Engineering coursework included control theory, signal processing, 
% antennas and transmission lines.  \\

{\bf Top 0.1\% students} -- PAU official exams \hfill {\em 2018} 
% \\ Generalitat de Catalunya \\

{\bf Scientific-Technological Baccalaureate with Honors} \hfill {\em 2016 -- 2018} 
\\ Maristes Sants-Les Corts, Barcelona \\
\end{rSection}
%----------------------------------------------------------------------------------------
%	TECHNICAL STRENGTHS SECTION
%----------------------------------------------------------------------------------------

% \begin{rSection}{Technical Skills}

% \begin{tabular}{ @{} >{\bfseries}l @{\hspace{6ex}} l }
% Language \ & C, C++ \\
% Software  \ & MS Office, Latex \\
% \end{tabular}

% \end{rSection}

%----------------------------------------------------------------------------------------
%	LANGUAGES
%----------------------------------------------------------------------------------------

\newpage

\begin{rSection}{Languages}
    \begin{tabular}{ @{} >{\bfseries}l @{\hspace{6ex}} l }
    Catalan, Spanish \ & Native \\
    English  \ & Proficient \\
    German  \ & Intermediate \\
    \end{tabular}
\end{rSection}

\begin{rSection}{Programming}
    \begin{tabular}{ @{} >{\bfseries}l @{\hspace{6ex}} l }
    Python \ & Machine learning, data analysis, computational physics. \vspace{-0.3em} \\
            & \textit{Pytorch, Pytorch Lightning, Weights \& Biases, ...} \\

    R  \ & Statistics, frequentist inference, statistical learning \\
    MATLAB  \ & Numerical methods for mathematics and physics, signal processing, linear  \vspace{-0.3em} \\
            & systems theory, optimization (IPM). \\
    AMPL \ & LP, IP and MILP optimization, stochastic programming. \\
    Stan \ & Bayesian analysis. \\
    Scala \ & FOOP, Spark RDDs. \\
    SAS \ & Statistical data analysis. \\
    Fortran \ & MD and MC simulations, VMD visualization. \\
    C, C\texttt{++} \ & Analog and digital circuit control. \\
    \end{tabular}
\end{rSection}
   
%----------------------------------------------------------------------------------------
%	EXAMPLE SECTION
%----------------------------------------------------------------------------------------

\begin{rSection}{Other projects}
% {\bf Erasmus ULISSES Project Ideathon} -- UL (Lisbon) \hfill \vspace{0.5em} {\em Jul. 2023} 
% \\ Three-week intensive ideathon aimed at addressing an ocean sustainability
% challenge. The project required a previous semester e-learning
% phase and a thorough investigation of the state-of-the art in marine
% sciences. Ideathon involved satellite data collection, solution design, viability assessment and product showcasing processes. \\
% % (Copernicus, IOOS, EMODnet...)

% {\bf Modelling and Design of a Paul Ion Trap} -- Computational Electromagnetism \hfill \vspace{0.5em} {\em 2021} 
% \\ Numerical simulation of the distribution of charges and the potential generated by a quadrupole electrode
% configuration using the method of moments, resulting in confined ion trajectories for harmonic potentials of varying frequencies.
% Starting with simpler models in 2D and 3D, the project provides optimal design parameters needed for
% confinement depending on the number and nature of the ions.\\

% {\bf Calcium-mediated regulation of astrocytes response} -- Computational biophysics \hfill \vspace{0.5em} {\em 2021} 
% \\ Analysis of the $Ca^{2+}$ concentration in brain astrocytes under different inositol triphosphate (IP3)
% signaling dynamics. Hodgkin-Huxley model was used to show spike-stimulated calcium oscillations due to
% the production of IP3 reaching threshold concentrations, and the continued influx of ions when neuron-coupled dynamics
% are considered, a malfunction that has been linked to the appearance of several mental diseases.
% \\

{\bf Erasmus ULISSES Ideathon} -- UL (Lisbon) \hfill  {\em Jul. 2023} \\
Three-week intensive ideathon aimed at addressing an ocean sustainability challenge.

{\bf Lasso and Bayes -- A demostration using real estate market data} \hfill {\em 2023} 
\\ Bayesian analysis. \\

{\bf Modelling and design of a Paul Ion Trap} \hfill {\em 2021} 
\\ Computational electromagnetism, finite element method, EM momentum method.\\

{\bf Calcium-mediated regulation of astrocytes response in the brain} \hfill {\em 2021} 
\\ Computational biophysics, dynamical systems modelling. \\

{\bf Design and implementation of a sound recorder, processor and player in the electronics laboratory.
TD-PSOLA and Phase Vocoder algorithms in a STM32 
% NUCLEO-F303RE
 microprocessor.} \hfill {\em 2021} 
\\ Electronics, signal processing, analog and digital circuits, PWM conversion. \\

{\bf Quantum Key Distribution in free-space-link communication systems. BB84 protocol implementation for earth-satellite communication.} \hfill {\em 2018} 
\\ Quantum physics, cryptography, free-space communication. \\
\end{rSection}

\end{document}